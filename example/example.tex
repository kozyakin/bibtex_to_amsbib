% !Mode:: "TeX:UTF-8"
%
% The WinEdt Pragma Comments below cause pdflatex + bibtexu commands
% to be executed when PDFTeXify is invoked.
%-------------------------------------------------------------------
% !PDFTeXify engine = pdflatex
% !BIB Exe = bibtexu
% !BIB Opt = -H -l ru -o ru
%-------------------------------------------------------------------
% You may also process this file in EVERY system with ARARA by running
% the following command in cmd/terminal: arara <this file>
%
% arara: clean: { extensions: [ aux, bbl, bcf, blg, out, log, run.xml, synctex, synctex.gz, toc ] }
% arara: pdflatex: {draft: yes, shell: yes}
% arara: biber if exists('bcf')
% arara: bibtexu: {options: ['-H', '-l', 'ru', '-o', 'ru']} if found('aux','bibdata') && !exists('bcf')
% arara: pdflatex: {shell: yes, synctex: yes}
% arara: --> until !found('log', '\\(?(R|r)e\\)?run (to get|LaTeX)')
% arara: clean: { extensions: [ aux, bbl, bcf, blg, out, log, run.xml, synctex, synctex.gz, toc ] }
%
\documentclass[a4paper]{article}
\usepackage[T1,T2A]{fontenc}
\usepackage[utf8]{inputenc}
\usepackage[english,russian]{babel}
\usepackage{a4wide}
\usepackage{amsmath,amssymb}
\usepackage[hyper]{amsbib}

\usepackage{fancyvrb}

\title{Преобразование библиографии из формата Bib\TeX{} в формат AMSBIB}

\author{В.\,С. Козякин\thanks{Институт проблем передачи информации им.
А.\,А.~Харкевича РАН, 127051 Москва, Большой Каретный переулок,~19, стр.~1,
e-mail: \href{mailto:kozyakin@iitp.ru}{kozyakin@iitp.ru}}}
%\date{}

\begin{document}
\maketitle

При подготовке рукописей для публикации в подавляющем числе (более 150-и) российских журналов математической направленности портал \href{https://www.mathnet.ru/}{Math-Net.Ru} рекомендует оформлять библиографию в стиле \href{https://www.mathnet.ru/poffice/amsbibpackage.phtml?wshow=amsbibpackage&option_lang=rus}{AMSBIB}.

В случае использования в библиографии ссылок на публикации в русскоязычных журналах, индексированных в Math-Net.Ru, особой проблемы при этом не возникает --- соответствующие библиографические записи в формате AMSBIB могут быть скопированы с соответствующих страниц публикаций на сайте \href{https://www.mathnet.ru/}{Math-Net.Ru}. Хуже обстоит дело со ссылками на англоязычные публикациии, большинство из которых не индексируется на сайте Math-Net.Ru, и для которых, соответственно, библиографическая информация в формате AMSBIB, как правило, отсутствует. В этом случае приходится вручную составлять соответствующие библиографические записи в формате AMSBIB, используя широко доступные (например, на сайте \href{https://mathscinet.ams.org/mrlookup}{MR Lookup}, на сайтах журналов или на многочисленных библиографических интернет-сервисах) соответствующие библиографические записи в формате Bib\TeX. 

К сожалению, между полями библиографических записей в форматах AMSBIB и Bib\TeX{} нет однозначного соответствия, поэтому процесс перевода записей одного формата в другой становится ``творческим''. Если такую процедуру требуется проделать для одной-двух публикаций, особых проблем не возникает. Но когда требуется перевести из формата Bib\TeX{} в AMSBIB достаточно большое количество библиографических записей (например, при подготовке обзора или монографии), задача становится малоприятной, не говоря уж о том, что ручной перевод чреват большим количеством ошибок, а также сильно зависит от ``творчества'' конкретного автора.

Чтобы упростить и унифицировать процесс преобразования библиографии из формата Bib\TeX{} в формат AMSBIB, мной были созданы стилевые файлы \texttt{amsbib.bst} и \texttt{amsbibs.bst}, осуществляющие такое преобразование автоматически. Причем, первый из этих стилевых файлов создает список библиографических записей AMSBIB в порядке цитирования публикаций в работе, а второй --- в алфавитном порядке. 

Пример такого преобразования приводится в листинге ниже, а его результат --- в конце данной работы: 
\begin{Verbatim}[frame=lines,fontsize=\small,label=Фрагмент tex-файла примера,framesep=12pt]
\documentclass[a4paper]{article}
\usepackage[T1,T2A]{fontenc}
\usepackage[utf8]{inputenc}
\usepackage[english,russian]{babel}
\usepackage{amsmath,amssymb}
\usepackage[hyper]{amsbib}
......................................
...дополнительные команды преамбулы...
......................................
\title{....}
\author{....}

\begin{document}
\maketitle
......................................
...........текст публикации...........
......................................
\nocite{*}

\bibliographystyle{amsbib}
\bibliography{example}
\end{document} 
\end{Verbatim}

При этом сама библиография (созданная с помощью пакета \texttt{amsbib.sty}) как вставляется в pdf-файл, создаваемый при трансляции tex-файла, так и помещается в файл \texttt{<имя файла>.bbl}, генерируемый при трансляции tex-файла.

Подчеркнем, что при этом как файл библиографии \texttt{.bib}, так и использующий его tex-файл должны быть в одной кодировке. Например, в данной работе использовалась кодировка \texttt{utf8}. При этом, в случае использования кодировок \texttt{cp866} или \texttt{cp1251}, для обработки библиографии должна применяться программа \texttt{bibtex8}, а при использования кодировки \texttt{utf8} --- программа \texttt{bibtexu}.

Предлагаемые стилевые файлы \texttt{amsbib.bst} и \texttt{amsbibs.bst} далеки от совершенства --- это лишь первая попытка в данном направлении. Поэтому \textbf{рекомендуется полученный в результате список библиографических записей в формате AMSBIB тщательно проверить и, при необходимости, откорректировать вручную.}

\sloppy Cтилевые файлы \texttt{amsbib.bst} и \texttt{amsbibs.bst} и файлы примеров \texttt{example.tex} и \texttt{\detokenize{example_en.tex}} могут быть загружены со страницы \href{https://github.com/kozyakin/kozyakin.github.io/tree/main/bibtex_to_amsbib}{BibTeX to AMSBIB} моего репозитария GitHub~Pages \href{https://kozyakin.github.io/}{kozyakin.github.io}. Необходимые для трансляции примеров файлы пакета AMSBIB (\texttt{amsbib.sty} + \texttt{*.pdf}) позаимствованы из \href{https://www.mathnet.ru/poffice/amsbib.zip}{amsbib.zip}.
\bigskip

Ниже приводится фрагмент базы данных \texttt{amsbib.bib} библиографии в формате Bib\TeX{}, использованной в данном примере:

\begin{Verbatim}[frame=lines,fontsize=\small,label=Фрагмент базы данных BibTeX amsbib.bib,framesep=12pt]
@ARTICLE{BKK:IEEETNN96,
 author        = "Bhaya, Amit and Kaszkurewicz, Eugenius and Kozyakin, V. S.",
 title         = "Existence and stability of a unique equilibrium in
                  continuous-valued discrete-time asynchronous {H}opfield
                  neural networks",
 journal       = "IEEE Trans. Neural Netw.",
 fjournal      = "IEEE Transactions on Neural Networks",
 year          = "1996",
 volume        = "7",
 number        = "3",
 pages         = "620--628",
 month         = may,
 issn          = "1045-9227",
 doi           = "10.1109/72.501720",
 url           = "https://ieeexplore.ieee.org/document/501720",
 language      = "english",
}

@ARTICLE{ChadKra:APM2:97,
 author        = "Ch{\k{a}}dzy{\'n}ski, Jacek and Krasi{\'n}ski, Tadeusz",
 title         = "A set on which the {{\L}}ojasiewicz exponent at infinity is
                  attained",
 journal       = "Ann. Polon. Math.",
 fjournal      = "Annales Polonici Mathematici",
 year          = "1997",
 volume        = "67",
 number        = "2",
 pages         = "191--197",
 eprinttype    = "arXiv",
 eprint        = "math/9802064",
 coden         = "APNMA4",
 issn          = "0066-2216",
 mrclass       = "14E05",
 mrnumber      = "1460600 (98j:14013)",
 mrreviewer    = "Zbigniew Jelonek",
 language      = "english",
}

.......................................................

@BOOK{AizGant:r,
 author        = "Айзерман, М. А. and Гантмахер, Ф. Р.",
 title         = "Абсолютная устойчивость регулируемых систем",
 publisher     = "Изд-во АН СССР",
 address       = "М.",
 year          = "1963",
 pagetotal     = "140",
 language      = "russian",
}

@ARTICLE{Anosov:PSIM67:r,
 author        = "Аносов, Д. В.",
 title         = "Геодезические потоки на замкнутых римановых многообразиях
                  отрицательной кривизны",
 journal       = "Тр. МИАН",
 fjournal      = "Труды Математического института имени В. А. Стеклова",
 year          = "1967",
 volume        = "90",
 pages         = "3--209",
 url           = "https://mi.mathnet.ru/tm2795",
 language      = "russian",
}

........................................................
\end{Verbatim}
\bigskip

Далее приводится фрагмент файла \texttt{example.bbl}, сгенерированного в результате конвертации и содержащего базу данных библиографии в формате AMSBIB:

\begin{Verbatim}[frame=lines,fontsize=\small,label=Фрагмент полученного файла example.bbl базы данных AMSBIB,framesep=12pt]
\begin{thebibliography}{10}
% \bib, bibdiv, biblist are defined by the amsrefs package.

\Bibitem{BKK:IEEETNN96}
\by A.~Bhaya, E.~Kaszkurewicz, V.~S.~Kozyakin
\paper  Existence and stability of a unique equilibrium in continuous-valued
  discrete-time asynchronous {H}opfield neural networks
\jour IEEE Trans. Neural Netw.
\yr 1996
\vol 7
\issue 3
\monthissue May
\pages 620--628
\crossref{https://dx.doi.org/10.1109/72.501720}
\elink{\url{ https://ieeexplore.ieee.org/document/501720}}

\Bibitem{ChadKra:APM2:97}
\by J.~Ch{\k{a}}dzy{\'n}ski, T.~Krasi{\'n}ski
\paper  A set on which the {{\L}}ojasiewicz exponent at infinity is attained
\jour Ann. Polon. Math.
\yr 1997
\vol 67
\issue 2
\pages 191--197
\arxiv \href{http://arXiv.org/abs/math/9802064}{\allowbreak
  math/9802064}\miscnote
\mathscinet{https://www.ams.org/mathscinet-getitem?mr=1460600}

.......................................................

\RBibitem{AizGant:r}
\by М.~А.~Айзерман, Ф.~Р.~Гантмахер
\book Абсолютная устойчивость регулируемых
  систем
\yr 1963
\publ Изд-во АН СССР
\publaddr М.
\totalpages 140

\RBibitem{Anosov:PSIM67:r}
\by Д.~В.~Аносов
\paper  Геодезические потоки на замкнутых
  римановых многообразиях отрицательной
  кривизны
\jour Тр. МИАН
\yr 1967
\vol 90
\pages 3--209
\mathnet{https://mi.mathnet.ru/tm2795}

........................................................

\end{thebibliography}
\end{Verbatim}

\fussy\nocite{*}

\bibliographystyle{amsbib}
\bibliography{example}
\end{document} 